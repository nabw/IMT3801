\documentclass{article}
\usepackage[utf8]{inputenc}
\usepackage{amsmath, amsthm, amssymb, mathpazo, isomath, mathtools}
\usepackage{subcaption,graphicx,pgfplots}
\usepackage{fullpage}
\usepackage{booktabs}
\usepackage{hyperref}
\usepackage{algorithm, algorithmic}
\usepackage{mathtools}
\usepackage{todonotes}

\newcommand{\example}[1]{\todo[inline,color=green!30!white]{\textbf{Example:} #1}}

\title{Tarea 1}
%\author{Nicol\'as A Barnafi\thanks{Instituto de Ingeniería Biológica y Médica, Pontificia Universidad Católica de Chile, Chile}, Axel Osses\thanks{Departamento de Ingeniería Matemática, Universidad de Chile, Chile}}
\author{Nicol\'as A Barnafi}
%\date{}

\renewcommand{\vec}{\vectorsym}
\newcommand{\mat}{\matrixsym}
\newcommand{\ten}{\tensorsym}
\DeclareMathOperator{\grad}{\nabla}
\DeclareMathOperator{\dive}{\text{div}}
\DeclareMathOperator{\curl}{\text{curl}}
\DeclareMathOperator{\tr}{\text{tr}}
\DeclareMathOperator{\sym}{\text{sym}}
\newtheorem{remark}{Remark}
\newtheorem{definition}{Definition}
\newcommand{\R}{\mathbb{R}}
\newcommand{\D}{\mathcal{D}}

\newcommand{\tin}{\text{in}}
\newcommand{\ton}{\text{on}}

\newtheorem{theorem}{Theorem}
\newtheorem{lemma}{Lemma}

\begin{document}

\maketitle

\begin{enumerate}
    \item El espacio de movimientos rígidos, dado por el kernel del gradiente simétrico $\varepsilon(u) = \frac 1 2 \left( \grad u + (\grad u)^T \right)$ está dado por el espacio generado por las funciones $(1,0)$, $(0,1)$, y $(-y, x)$. Definimos dicho espacio como $\mathbb{RM}$. Encuentre la proyección a $\mathbb{RM}$ de la functión 
        $$ f(x,y) = \sin(x) + \cos(y) $$
    en $\mathbb{RM}$ con respecto a los espacios $L^2(\Omega)$ y $H^1(\Omega)$, con $\Omega=[0,1]^2$.
    \item Considere la siguiente función: 
        $$ f(x) = \begin{cases}
                        1 & \text{$x$ in $(0,1)$} \\
                        0 & \text{elsewhere}
                    \end{cases}. $$
    Muestre que $f$ está en $L^\infty(\R)$ pero no en $W^{1,\infty}(\R)$. Extienda la demostración para alguna función discontinua $\R^2$. 
    \item Encuentre una generalización de la desigualdad de Hölder para productos de $n$ funciones: $\| f_1 \hdots f_n \|_{L^1(\Omega)} \leq \| f_1\|_{L^{a_1}(\Omega)} \hdots \| f_n \|_{L^{a_n}(\Omega)}$. Qué condiciones deben cumplir los exponentes $a_i$? 
    \item Demuestre las tres identidades de integración por partes mostradas en los apuntes del curso.
    \item Se define la derivada de Gateaux de un funcional $\Pi: V \to \R$ en $u$ y en dirección $v$ como
        $$ d\Pi(u)[v] \coloneqq \frac{d}{d\epsilon}\left.\left(\Pi(u + \epsilon v) \right)\right|_{\epsilon=0}. $$
    Muestre que dada una forma bilineal simétrica $a: V\times V \to \R$ y un funcional lineal $L:V\to \R$, se tiene que las condiciones de primer orden del funcional cuadrático $\Pi(u) = \frac 1 2 a(u,u) + L(u)$ corresponden a la ecuación 
        $$ a(u,v) = L(v) \qquad\forall v \in V. $$
    Muestre además que los puntos $u$ en $V$ que satisfacen dicha ecuación son efectivamente mínimos del funcional cuadrático. Finalmente, muestre que esto caracteriza a las funciones armónicas ($-\Delta u=f$) como aquellas que minimizan la semi norma $H^1$. 
    \item Considere el laplaciano biarmónico $\Delta^2 u$, definido como aplicar dos veces el Laplaciano $\Delta = \sum_i \partial_i^2$. Encuentre la formulación variacional de este problema, y sus correspondientes condiciones de Dirichlet y de Neumann. 
    \item Considere el tensor de Hooke, tensor del cuarto orden, cuya acción está dada por 
        $$ \mathbb{C}_\text{Hooke}\tau = \lambda \tr \tau \ten I + 2 \mu \tau, $$
    donde $\lambda, \mu$ se conocen como parámetros de Lamé. Considere además el gradiente simétrico $\varepsilon(u) = \frac 1 2 \left( \grad u + (\grad u)^T \right)$. Encuentre la formulación débil del problema de elasticidad lineal, cuya forma fuerte está dada por 
        $$ - \dive \mathbb C_\text{Hooke}\varepsilon(\vec u) = \vec f $$
    para alguna fuerza externa $f$. Determine la regularidad necesaria de $f$, así como también las condiciones de borde sugeridas por la integración por partes. Finalmente, demuestre que dada una matriz arbitraria $A$ y una simétrica $S$, se tiene que  
        $$ A : S = \sym(A) : S, $$
    donde $\sym(A) = \frac 1 2\left(A + A^T\right)$, para encontrar una formulación variacional donde la forma bilineal sea simétrica. 
\end{enumerate}
\end{document}

