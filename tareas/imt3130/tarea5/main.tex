\documentclass{article}
\usepackage[utf8]{inputenc}
\usepackage{amsmath, amsthm, amssymb, mathpazo, isomath, mathtools}
\usepackage{subcaption,graphicx,pgfplots}
\usepackage{fullpage}
\usepackage{booktabs}
\usepackage{hyperref}
\usepackage{algorithm, algorithmic}
\usepackage{mathtools}
\usepackage{todonotes}

\title{Tarea 5}
%\author{Nicol\'as A Barnafi\thanks{Instituto de Ingeniería Biológica y Médica, Pontificia Universidad Católica de Chile, Chile}, Axel Osses\thanks{Departamento de Ingeniería Matemática, Universidad de Chile, Chile}}
%\author{Nicol\'as A Barnafi}
\date{}

\renewcommand{\vec}{\vectorsym}
\newcommand{\mat}{\matrixsym}
\newcommand{\ten}{\tensorsym}
\DeclareMathOperator{\grad}{\nabla}
\DeclareMathOperator{\dive}{\text{div}}
\DeclareMathOperator{\curl}{\text{curl}}
\DeclareMathOperator{\tr}{\text{tr}}
\DeclareMathOperator{\sym}{\text{sym}}
\newtheorem{remark}{Remark}
\newtheorem{definition}{Definition}
\newcommand{\R}{\mathbb{R}}
\newcommand{\D}{\mathcal{D}}

\newcommand{\tin}{\text{in}}
\newcommand{\ton}{\text{on}}

\newtheorem{theorem}{Theorem}
\newtheorem{lemma}{Lemma}
\newcommand{\pts}[1]{[{\bf #1 puntos}] }

\begin{document}

\maketitle
\hfill \textbf{Fecha de entrega: 23:59 del 13/06/2025}
 
\todo[inline,color=white!90!black]{\textbf{Instrucciones: } La tarea debe ser entregada de manera individual en un informe en formato .pdf a través del buzón habilitado en la plataforma Canvas, donde deben mostrar también el código desarrollado. Para su conveniencia, pueden entregar las tareas en un Jupyter Notebook, de modo que sea más cómodo mostrar el código. En cualquier caso, se debe entregar un único archivo como respuesta a la tarea. La política de atrasos será: se calculará un factor lineal que vale 1 a la hora de entrega y 0 48 horas después. Esto multiplicará su puntaje obtenido. Pueden usar ChatGPT u otros modelos solo a conciencia. El uso de salidas de GPT sin su debida comprensión será severamente sancionado. }

\begin{enumerate}
    \item Definir una deformación y calcular todos los vectores y tensores relevantes. Graficar en python caso 2D. 
    \item Demostrar que invariantes son invariantes (Tarea 2 Federico)
    \item demostrar con Caley Hamilton d J /d F = cof(F)
    \item Mostrar que si el Piola depende del strain rate, se tiene Navier Stokes
    \item Definir energía Neo-Hookeana, calcular Piola y escribir ecuación de conservación de momentum
    \item Construir ecuación ADR desde conservación de masa bajo ciertos supuestos
    \item Mostrar que linealizar en E da elasticidad lineal. Encontrar parámetros de Lamé. 

\end{enumerate}

\todo[inline,color=white!90!black]{\textbf{Nota: } Abriremos un foro en Canvas para revisar cualquier typo y/o error que haya en el enunciado.}
\end{document}

