\documentclass{article}
\usepackage[utf8]{inputenc}
\usepackage{amsmath, amsthm, amssymb, mathpazo, isomath, mathtools}
\usepackage{subcaption,graphicx,pgfplots}
\usepackage{fullpage}
\usepackage{booktabs}
\usepackage{hyperref}
\usepackage{algorithm, algorithmic}
\usepackage{mathtools}
\usepackage{todonotes}

\title{Examen}
%\author{Nicol\'as A Barnafi\thanks{Instituto de Ingeniería Biológica y Médica, Pontificia Universidad Católica de Chile, Chile}, Axel Osses\thanks{Departamento de Ingeniería Matemática, Universidad de Chile, Chile}}
%\author{Nicol\'as A Barnafi}
\date{}

\renewcommand{\vec}{\vectorsym}
\newcommand{\mat}{\matrixsym}
\newcommand{\ten}{\tensorsym}
\DeclareMathOperator{\grad}{\nabla}
\DeclareMathOperator{\dive}{\text{div}}
\DeclareMathOperator{\curl}{\text{curl}}
\DeclareMathOperator{\tr}{\text{tr}}
\DeclareMathOperator{\sym}{\text{sym}}
\newtheorem{remark}{Remark}
\newtheorem{definition}{Definition}
\newcommand{\R}{\mathbb{R}}
\newcommand{\D}{\mathcal{D}}

\newcommand{\tin}{\text{in}}
\newcommand{\ton}{\text{on}}

\newtheorem{theorem}{Theorem}
\newtheorem{lemma}{Lemma}
\newcommand{\pts}[1]{[{\bf #1 puntos}] }

\begin{document}

\maketitle
\hfill \textbf{Fecha de entrega: 23:59 del 30/06/2025}
 
\todo[inline,color=white!90!black]{\textbf{Instrucciones: } La tarea debe ser entregada de manera individual en un informe en formato .pdf a través del buzón habilitado en la plataforma Canvas, donde deben mostrar también el código desarrollado. Para su conveniencia, pueden entregar las tareas en un Jupyter Notebook, de modo que sea más cómodo mostrar el código. En cualquier caso, se debe entregar un único archivo como respuesta a la tarea. Entregas atrasadas tienen un 1.0 automáticamente. Pueden usar ChatGPT u otros modelos solo a conciencia. El uso de salidas de GPT sin su debida comprensión será severamente sancionado. }

\begin{enumerate}

    \item Considere el problema ADR con condiciones de borde homogéneas:
            $$ \begin{aligned}
                -\mu \Delta u + \vec b \cdot \grad u + c u &= f \qquad \Omega \\
                u &= 0 \qquad \partial\Omega.
            \end{aligned}$$
            Asumiendo que $\mu,\vec b,c$ son constantes, considere la discretización de este problema en el dominio $\Omega = (0,1)$. 
            \begin{itemize}
                \item Muestre que este problema tiene una única solución continua. Hint: Use Lax-Milgram y concluya la regularidad con teoremas de inmersión)
                \item Describa cómo aproximar este problema con diferencias finitas 
                \item Describa cómo aproximar este problema con elementos finitos (de primer orden)
                \item Encuentre la expresión algebraica de las matrices discretas para elementos finitos y diferencias finitas. Existe algún rango de parámetros donde las discretizaciones resulten en el mismo problema? 
                \item Explique las garantías teóricas que tienen FEM y FD, y compárelas. Qué le parece más conveniente? 
            \end{itemize}

    \item Algo de contimech
\end{enumerate}

\todo[inline,color=white!90!black]{\textbf{Nota: } Abriremos un foro en Canvas para revisar cualquier typo y/o error que haya en el enunciado.}
\end{document}

